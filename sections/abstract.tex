% \section{Abstract}

The Internet hosts various amount of web applications, services, and even more users.
These users, while using an application, often want to access their data that resides in another, third-party application, i.e., when a user uses an email application and wants to access his contacts that exist in another application, how will the email application eventually get his contacts?
In the old-fashioned way, the user had to hand over the credentials of his third-party application account to the application he eventually wants to get his data.
% but the user gave access to all of his data in the 3rd party application, including his password.
This was insecure for the user, the user gave access to all of his data in the 3rd party application, including his password.
With that method, the application that took his credentials could store them in plain text, the database could be hacked, and to revoke access, there was only one option; to change his password. With Single Sign-on (SSO), this problem no longer exists, as users can authorize the application to access a specific set of data residing elsewhere without disclosing passwords.
% , and give them access to a specific set of data.
Today there are many protocols and frameworks to achieve SSO, for users to be able to authorize and authenticate using third-party applications.
In this thesis, we examine SSO between Internet services using the OAuth 2.0~\cite{OAUTH} and OpenID Connect~\cite{OIDC}  protocols, by implementing example web-based applications in Python using the Django~\cite{Django} framework. We also extend one of the example applications to use a remote LDAP~\cite{LDAP} server as a user directory instead of a local user database.
