\section{Conclusion}
We managed to enable both OAuth and OIDC protocols to authorize and authenticate between Django applications. We built two Provider applications with Django using the Django-OAuth-Toolkit, acting as both Resource Servers and Authentication Servers, one for OAuth and one for OIDC. With Django we also created the Client, an application for users to login via the Providers, by creating 2 custom backends (one for OAuth and one for OIDC) which override the base OAuth class provided by django-social-auth, which allowed us to authenticate and authorize from the Client application. With the auth-ldap package, we added LDAP authentication functionality in the OIDC implementation of the Provider using FORTH's LDAP server. Code is available at \href{https://github.com/CARV-ICS-FORTH/django-oauth2-oidc-example}{https://github.com/CARV-ICS-FORTH/django-oauth2-oidc-example.}


\begin{acks}
% I would like to thank Prof. Agelos Bilas for allowing me to implement my thesis in his lab, and Mr. Antonis Xazapis for providing guidance, giving valuable comments and suggestions during this work
I would like to thank all the members of the CARV lab at FORTH for their help and support, especially Antony Chazapis for providing guidance, giving valuable comments and suggestions during this work.
%\hl{Probably just thank the members of the CARV lab at FORTH for their help, etc.}
\end{acks}
