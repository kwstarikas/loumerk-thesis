\section{Conclusion}
We managed to enable both OAuth 2.0 and OIDC protocols to authorize and authenticate between Django applications. We built two applications with Django, and made them Providers with Django-OAuth-Toolkit, acting as Resource Servers and Authentication Servers, one for OAuth and one for OIDC. With Django we create the Client, an application for users to login via the Providers, made possible by creating 2 custom backends (one for OAuth2.0 and one for OIDC) which override the base OAuth class that the django-social-auth provides and we managed to authenticate and authorize from the Client application. With the auth-ldap package, we added LDAP authentication functionality to the OIDC implementation of the provider using FORTH's LDAP server. Code is available at \href{https://github.com/CARV-ICS-FORTH/django-oauth2-oidc-example}{https://github.com/CARV-ICS-FORTH/django-oauth2-oidc-example.}


\begin{acks}
% I would like to thank Prof. Agelos Bilas for allowing me to implement my thesis in his lab, and Mr. Antonis Xazapis for providing guidance, giving valuable comments and suggestions during this work
\hl{Probably just thank the members of the CARV lab at FORTH for their help, etc.}
\end{acks}
